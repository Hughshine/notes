\documentclass[11pt,largemargins]{homework}


\newcommand{\hwname}{李煦阳 DZ21330015}
\newcommand{\hwemail}{njulixuy@163.com}
\newcommand{\hwtype}{作业1}
\newcommand{\hwnum}{}
\newcommand{\hwclass}{计算复杂性}
\newcommand{\hwlecture}{}
\newcommand{\hwsection}{}
\renewcommand{\qed}{$\hfill\blacksquare$}

% This is just used to generate filler content. You don't need it in an actual
% homework!
\usepackage{lipsum}
\usepackage{ctex}

\begin{document}
\maketitle

\question*{5.13, VC-Dimension}

VC-d的定义:$\mathcal{S}=\{S_1,S_2,\dots,S_m\}$是有限集$U$的子集集合(表示某类模型所有可以产生的二分),集合$\mathcal{S}$的VC-d指可以被该模型二分的最大集合的大小(这个集合,无论我们如何划分它,这个划分都可以被某个实例模型产生的二分描述)(the size of largest set $X\subseteq U$, such that $\forall X'\subseteq X, $\\$\exists i, S_i \cap X = X'$)。% 不清楚U是有限集有什么影响,可以就当作数据量吧

\begin{alphaparts}
  \questionpart
  要证明所有的\textbf{NP}问题($L\in \textbf{NP}$)都可以规约至停机问题。\newline
  已知$\forall L\in \textbf{NP}.\;\exists M,p.\;\forall x.\;x\in L \Leftrightarrow \exists u\in \{0,1\}^{p(|x|)}.\;M(x,u)=1.$\newline
  令规约函数$f(x) = \langle\langle\alpha\rangle, \langle\beta, \theta, x\rangle\rangle$,其中$M_\beta = M$, $\theta$为p的编码,$M_\alpha$重复遍历$u\in\{0,1\}^{p|x|}$,在$M_{\beta}(x, u) = 1$时停机,否则不停机。易知其复杂度为常数,是一个双射函数。\\
  基于$M$构造\texttt{HALT}问题:
  输入为$\langle\langle\alpha\rangle,\langle\beta, \theta, x\rangle\rangle$, 其对应的语言为$L'$\\
  现证$x\in{L}\Leftrightarrow f(x)\in {L'}$.
  \begin{enumerate}
    \item $x\in L\Rightarrow f(x)\in L'$\\
          由构造$L'$的方式可知,对于$x\in L$,$f(x)$在\texttt{HALT}上会停机,所以$f(x)\in L'$。
    \item $f(x)\in L'\Rightarrow x\in L$\\ 
          反证,若存在$f(x)\in L'$且$x\not\in L$,则$x$在构造的图灵机上不会停机,所以\texttt{HALT}$(f(x))$不为$0$,所以$f(x)\not\in L'$
  \end{enumerate}\qed
  \questionpart
  \texttt{HALT}问题不是\textbf{NP}问题。\\
  易知\texttt{HALT}问题是不可判定问题(并不存在一种算法可以描述\texttt{HALT}问题)(将\texttt{HALT}问题带入自身可证)。只需证所有\textbf{NP}问题都是可判定的(即可以找到一个通用算法),便可说明\texttt{HALT}不是\textbf{NP}。

  对于每个\textbf{NP}问题,已知$M$,\textbf{P},对于一个输入$x$,我们可以对解空间进行$EXP(p(|x|)$次枚举寻找certificate,并在多项式时间内演算每个可能解的真假(总复杂度为$EXP(p(|x|))$),所以\textbf{NP}是可判定的。
  
  \qed
\end{alphaparts}

\end{document}
