\documentclass[11pt,largemargins]{homework}


\newcommand{\hwname}{李煦阳 DZ21330015}
\newcommand{\hwemail}{njulixuy@163.com}
\newcommand{\hwtype}{作业1}
\newcommand{\hwnum}{}
\newcommand{\hwclass}{计算复杂性}
\newcommand{\hwlecture}{}
\newcommand{\hwsection}{}
\renewcommand{\qed}{$\hfill\blacksquare$}

% This is just used to generate filler content. You don't need it in an actual
% homework!
\usepackage{lipsum}
\usepackage{ctex}

\begin{document}
\maketitle

\question*{2.8}

\begin{alphaparts}
  \questionpart
  要证明所有的\textbf{NP}问题($L\in \textbf{NP}$)都可以规约至停机问题。\newline
  已知$\forall L\in \textbf{NP}.\;\exists M,p.\;\forall x.\;x\in L \Leftrightarrow \exists u\in \{0,1\}^{p(|x|)}.\;M(x,u)=1.$\newline
  令规约函数$f(x) = \langle\langle\alpha\rangle, \langle\beta, \theta, x\rangle\rangle$,其中$M_\beta = M$, $\theta$为p的编码,$M_\alpha$重复遍历$u\in\{0,1\}^{p|x|}$,在$M_{\beta}(x, u) = 1$时停机,否则不停机。易知其复杂度为常数,是一个双射函数。\\
  基于$M$构造\texttt{HALT}问题:
  输入为$\langle\langle\alpha\rangle,\langle\beta, \theta, x\rangle\rangle$, 其对应的语言为$L'$\\
  现证$x\in{L}\Leftrightarrow f(x)\in {L'}$.
  \begin{enumerate}
    \item $x\in L\Rightarrow f(x)\in L'$\\
          由构造$L'$的方式可知,对于$x\in L$,$f(x)$在\texttt{HALT}上会停机,所以$f(x)\in L'$。
    \item $f(x)\in L'\Rightarrow x\in L$\\ 
          反证,若存在$f(x)\in L'$且$x\not\in L$,则$x$在构造的图灵机上不会停机,所以\texttt{HALT}$(f(x))$不为$0$,所以$f(x)\not\in L'$
  \end{enumerate}\qed
  \questionpart
  \texttt{HALT}问题不是\textbf{NP}问题。\\
  易知\texttt{HALT}问题是不可判定问题(并不存在一种算法可以描述\texttt{HALT}问题)(将\texttt{HALT}问题带入自身可证)。只需证所有\textbf{NP}问题都是可判定的(即可以找到一个通用算法),便可说明\texttt{HALT}不是\textbf{NP}。

  对于每个\textbf{NP}问题,已知$M$,\textbf{P},对于一个输入$x$,我们可以对解空间进行$EXP(p(|x|)$次枚举寻找certificate,并在多项式时间内演算每个可能解的真假(总复杂度为$EXP(p(|x|))$),所以\textbf{NP}是可判定的。
  
  \qed
\end{alphaparts}

\question*{2.15}
\begin{alphaparts}
  \questionpart 证明\texttt{CLIQUE}问题(某图的$K$个顶点的全连接子图是否存在问题)是\textbf{NPC}的,
  只需证独立集问题与\texttt{CLIQUE}问题可以相互规约。
  
  对于$\texttt{INDSET}(V,E,K)$,构造$\texttt{CLIQUE}(V,\overline{E},K)$,其中$\overline{E} = \{(v_1, v_2)\;|\; (v_1, v_2)\not\in E\}$. 规约函数显然是多项式时间的。正确性在于,连接与独立(不连接)是显然对偶的。
  
  类似地,\texttt{CLIQUE}问题也可规约成\texttt{INDSET}问题。所以\texttt{CLIQUE}是\textbf{NPC}的。
  \qed
  \questionpart 证明顶点覆盖问题(某图是否存在大小为$K$的顶点子集,使得图中的每一条边至少有一个顶点落于其中)是\textbf{NPC}的。
  只需证明独立集问题可以与顶点覆盖问题相互规约。

  首先证明顶点覆盖$\texttt{VC}(V,E,K)$问题可以规约至独立集问题$\texttt{INDSET}(V,$ $E,|V|-K)$。令找到的独立集为$S$(可以存在多于$|V|-K$大小的独立集,但只取到$|V|-K$),$|S| = |V|-K$,我们证明$S'= V-S$一定是个顶点覆盖,其大小为$K$. 

  假设$S'$不是顶点覆盖,那么$\exists v_1, v_2\in V-S' = S.\; (v_1, v_2)\in E$,这与$S$是独立集矛盾。

  其次证明独立集问题$\texttt{INDSET}(V,E,K)$可以规约至顶点覆盖问题$VC(V,E,|V|-K)$。令找到的顶点覆盖集合为$S$,$|S|=|V|-K$,我们证明$S' = V-S$一定是独立集。

  假设$S'$不是独立集,则$\exists v_1, v_2 \in V-S' = S.\; (v_1, v_2)\in E$,这意味着$S$不是顶点覆盖,矛盾。
  \qed
\end{alphaparts}

\question*{2.23}
证明$\textbf{P}\subseteq \textbf{NP}\cap \textbf{coNP}$.

课上已证$\textbf{P}\subseteq \textbf{NP}$(构造\textbf{NP}问题,令certificate为空,于是\textbf{NP}的$M$就是\textbf{P}的$M$. 显然保证多项式时间),只需证$\textbf{P}\subseteq \textbf{coNP}$。

构造方式类似,只需额外令$M'(x, u) = M(x)$即可,即忽略certificate。
\qed 

\question*{2.24}
证明两个\textbf{coNP}的定义等价,\\即证对于$L$,$\overline{L} \in \textbf{NP} \Leftrightarrow \exists M,p.\;\forall u^{p(|x|)}.\; M(x, u) = 1$.

令$L$在\textbf{NP}语言中的图灵机与多项式分别为$M'$,$p'$.

已知$x\in\overline{L} \Leftrightarrow x\not\in L \Leftrightarrow \;\exists u'.\; M'(x, u')=1 $,取否可得,$x\in L \Leftrightarrow \neg \exists u'.\; M'(x, u')=1 \Leftrightarrow \forall u'.\; \texttt{flip}(M'(x, u'))=1$. 也就是说,由基于2.19定义的语言,它构造地满足2.20的定义,其中$M = flip \cdot M'$与$p = p'$.

由于一直在用等价推导,反方向的证明也是类似的。
\qed

\question*{2.25}
即证$\textbf{P} = \textbf{NP}\rightarrow \textbf{NP}=\textbf{coNP}$.
对于$L \in \textbf{P}$, 若可以证明$\overline{L} \in \textbf{P}$,根据定义2.19(补集在NP中的语言\textbf{coNP}的),并且由于$\textbf{P} = \textbf{NP}$,那么$\textbf{NP} = \textbf{coNP}$. 下面证明$L\in \textbf{P}\rightarrow \overline{L}\in \textbf{P}$.

对于$L \in \textbf{P}$,我们有图灵机$M$,$M(x) = 1\Leftrightarrow x\in L$.
也即$M(x) = 1\Leftrightarrow x\not\in \overline{L}$,
取否,得$\texttt{flip}(M(x)) = 1 \Leftrightarrow x\in\overline{L}$. \texttt{flip}显然不影响时间复杂度。
也就是说,对于$\overline{L}$,可以找到多项时间的$M' = \texttt{flip}\cdot M$,使得$x\in\overline{L}\Leftrightarrow M'(x)=1$. \textbf{P}中语言的补集仍属于\textbf{P}得证,进而,$\textbf{NP}=\textbf{coNP}$得证。
\qed

\question*{2.30, Berman's Theorem}
证明若$SAT$可以规约至一元问题$L$,那么$SAT$可以找到一种多项式时间算法(剪枝算法)。

令$SAT$到$L$的多项式时间规约为$f$,公式$\varphi$有k个自由变量,长度为$n$。$\varphi$的可满足性等价于两个有$k-1$个自由变量的子公式的可满足性(将一个自由变量分别取$0$与$1$),两个子公式的长度也为$n$. 若将$k$个变量全部拆解,则$\varphi$的可满足性变为$2^k$个公式的可满足性(一个树状搜索问题)。但是借助$f$,可以在拆解过程中不断剪枝,不必计算全部$2^k$的公式的值。

由于$f$是多项式时间的(设为$p(\cdot)$),那么$f(x)$的长度一定在$p(|x|)$内。在前述的问题拆解过程(每一次拆解使公式数增加1),每当公式集合达到$p(|x|)+1$,对每个公式$\varphi$求解$f(\varphi)$,根据鸽笼原理,至少有一个公式不是一元的或者与已有的一元公式重复,我们可以用线性时间(也就是多项式时间)筛去(剪枝掉)这部分重复的或者非一元的公式。最终,只需在剩余的$p(|x|)$大小的公式集合验算可满足性。

但还需要说明剪枝次数也是多项式时间的。公式集中每一个公式的自由变量个数都小于等于$k$,意味着每个公式最多引发$k$次剪枝,即一共最多引发$k\cdot p(|x|)$次剪枝,$k$与$n$是多项式关系的,所以总复杂度仍是多项式时间的。

\qed
\end{document}
