\documentclass[UTF8, 8pt, a4paper ]{ctexart}
\usepackage[utf8]{inputenc}
\usepackage[left = 15mm, right = 15mm, top = 20mm, bottom= 20mm]{geometry}

\usepackage{ksafe-lin-proof}
\newcommand{\absHistSet}[1]{\mathcal{S}\llbracket #1 \rrbracket}

\usetikzlibrary{decorations.pathmorphing}
\newcommand\xrsquigarrow[1]{%
	\mathrel{%
		\begin{tikzpicture}[%
			baseline={(current bounding box.south)}
			]
			\node[%
			,inner sep=.44ex
			,align=center
			] (tmp) {$\scriptstyle #1$};
			\path[%
			,draw,<-
			,decorate,decoration={%
				,zigzag
				,amplitude=0.7pt
				,segment length=1.2mm,pre length=3.5pt
			}
			]
			(tmp.south east) -- (tmp.south west);
			\path[draw] ($ (tmp.south west)+(0,-0.7mm) $)--($ (tmp.south west)+(0,0.7mm) $);
			
		\end{tikzpicture}
	}
}

\newcommand{\circline}{
	\mathrel{%
		\raisebox{0.15mm}{
		\begin{tikzpicture}
%			[%
%			baseline={(current bounding box.south)}
%			]
			\draw[line width=0.35pt] (0,0) circle (2pt);
			\draw[line width=0.35pt] (-1.2mm, 0)--(1.2mm, 0);
		\end{tikzpicture}\ }
	}
}

%\newcommand{\clientSet}{\mathcal{C}}
%\newcommand{\replicaSet}{\mathcal{R}}
%\newcommand{\faultyClients}{\mathfrak{C}}
%\newcommand{\faultyReplica}{\mathfrak{R}}

\newcommand{\clientSet}{\mathcal{C}_{\textit{ok}}}
\newcommand{\replicaSet}{\mathcal{R}_{\textit{ok}}}
\newcommand{\faultyClients}{\mathcal{C}_{\textit{bad}}}
\newcommand{\faultyReplica}{\mathcal{R}_{\textit{bad}}}

\newcommand{\inbuf}{\textit{Inbuf}\hspace{0.7mm}}
\newcommand{\phymap}[1]{\ulcorner #1 \urcorner}
\newcommand{\addrSet}{\mathcal{A}}

\begin{document}
	\fontsize{8pt}{8pt}
	\setlength{\baselineskip}{14pt}
	
	
%	$ \circ $ $ \circline $
	
\begin{small}
		
%	$ (W,S) \longmcltstep{\ H\ }{t,P} (W,S) $
	
%	$ (W,S) \longmcltstep{\ aslhdajshdkajshdj\ }{t,P} (W,S) $
	
%	$ (W,S) \longmcltstep{\ \ \ }{e} (W,S) $




	\textbf{第一章的模型是在之前consensus模型的基础上加入了}
	\begin{itemize}
		\item 网络故障,消息丢失和复制:Msg-Lose和Msg-Dup语义
		\item 线程故障
		\begin{itemize}
			\item 良性错误crash:\\
			在每个线程的local state中指定read-only variable "\textit{faulty}" 用于记录该线程是否有故障,指定"\textit{role}"用于记录该线程是client还是replica。定义操作语义Glb-Thd-Crash,在程序执行的任何时刻,faulty为true的线程都可以跳转到skip。%若是faulty client,不仅要跳转,还要产生stop事件
			
			\item 拜占庭故障
			\begin{itemize}
				\item 表达arbitrary program. 定义基于$ \clientSet, \faultyClients, \replicaSet, \faultyReplica $(4个互不相交的集合,分别是correct client集,faulty client集,correct replica集,faulty replica集)的初始程序W
				\begin{itemize}
					\item 不限制在$ \faultyClients, \faultyReplica $集合中的线程的初始程序,并将其初始stack设为$ (\emptyset, \wildcard, \textbf{skip}) $(这样做是为了合乎操作语义,使他们可以执行底层send, recv原语)。
					\item 对应地,correct replica初始stack为$ (\emptyset, \wildcard, \textbf{skip}) $,且其必须执行协议指定的$ C_{serv} $程序. correct client的初始stack为空,它不能执行send,recv,只能通过protocol提供的函数接口访问shared variable
				\end{itemize}
				
				\item 建模faulty线程的"扮演"行为,根据"扮演"的定义,需要
				\begin{itemize}
					\item  建模“数字签名”:定义消息$ m\ (\textit{Msg}) $就是经过私钥签名后的消息内容

					\item  建模线程的"数字签名":在每个线程的local state中指定read-only thread id set variable "\textit{Keys}",用于记录该线程拥有哪些线程的私钥。在初始configuration中,faulty client都拥有所有faulty client的私钥,faulty replica拥有所有faulty replica的私钥,correct thread只有自己的私钥。
					
					\item  “数字签名”在消息发送和接收时扮演的角色:将send语句定义为send(tid, Dest, mc),语义是使用线程tid的签名,将内容为mc的消息发到Dest集合指定的节点中。并要求输入的tid必须属于Keys集合。使得faulty client和faulty replica可以通过输入不同tid扮演其他faulty线程\\
					此外,前面做consensus的文档中,线程id和收发消息的地址混在一起。这里为建模"扮演",需要区分开来(否则接收伪造签名的一方会把reply消息发送到被伪造的线程上去)。由此定义物理地址以与线程id区分。
				\end{itemize}
			\end{itemize}
		\end{itemize}
	\end{itemize}

	\textbf{第二章为了表达byznearizability}
	\begin{itemize}
		\item abstract machine 建模 arbitrary program. 在abstract machine中,所有client都必须通过object spec访问虚拟object, 故好client和坏client都不会对初始做限制。(除为了生成stop事件,坏client的初始程序的最后一句需要是\textbf{stop}外)
			
		\item abstract machine 建模 “扮演”。\\
			类似地,初始时给每个client指定的local read-only variable "\textit{faulty}"和"\textit{Keys}", 并将\textbf{fexec}语句修改为$ \textbf{fexec}(f, val, tid) $,该语句的语义是:以$ {tid} $线程的名义执行函数$ f $, 参数是$ val $. 任何client执行语句$ x \assign f(val) $, 都将跳到$ \textbf{fexec}(f, val, tid) $语句,$ tid $参数是非确定的$ \textit{Keys} $中的一个值。由于好client的$ \textit{Keys} $集合中只有自己,它只能以自己的名义执行,而坏client可以以人任何坏client的名义执行。
		
		\item abstract machine 建模 "余波"\\
			"余波"建模要求bad client在stop后仍可以产生有限数量的inv, res事件对。定义语句$ \textbf{aftermath}(H) $, 任何bad clients在产生stop事件(执行了stop语句或crash)后都跳转到该语句,$ H $是非确定的、有限长的、事件对结构的history. $ \textbf{aftermath}(H) $每执行一次,消耗$ H $开头的一个合法事件对,直至$ H $的开头两个事件对不合法或$ H $的长度小于2(将block)
		
		\item byznearizability中讨论的history可以是无限长的,需要对应co-inductive构造来表达\\
%		\red{一个问题是我在表达byznearizability时,尝试求infinite history的completion,我觉得这种做法可能是错的}
	\end{itemize}
	
	
	\pagebreak

	\section{Basic Technical Setting for Byzantine Objects}
	
	
	
	\newcommand{\auxFuncs}{\mathcal{F}_\textsf{aux}}
	
	\subsection{Syntax of the Programming Language}
	
		\begin{longtable}{rcl}
			\hline
				$ val $ & $ \in $ & $ \textit{Val} $ \\
				
				$ x, y, v...  $ & $ \in $ & \textit{PVar} \\ 
				
				$ (\!\textit{FuncName})\ f  $ & $ \in $ & $ \textit{String} $\\
				
%				$ (\!\textit{FuncSet})\ \mathcal{F} $ & $ \in $ & $ \mathscr{P}(\textit{FuncName}) $ \\
				
				$ (\!\textit{MsgTag})\ \textit{tag} $ & $ \in $ & $ \textit{String} $\\
				
				$ (\!\textit{MsgCont})\ mc $ & ::= & $ [\textit{tag}, val, ..., val] $ \\
				
				$ (\!\textit{ThreadID}) \ t, i, j $ & $ \in $ & $ \textit{Nat}^{+} $ \\
				
				$ (\!\textit{PhyAddr})\ pa $ & $ \in $ & $ \textit{Nat}^{+} $ \\
				
				$ (\!\textit{ThreadIDSet})\ \mathcal{C}, \mathcal{R} $ & $ ::= $ & $ \{ t_1, ..., t_n \} $ \\
				
				$ (\!\textit{AddrSet})\ \addrSet $ & $ ::= $ & $ \{ pa_1, ..., pa_n \} $ \\
				\hline
				\multicolumn{3}{c}{\textbf{Figure 1.} Basic Types} \\
		\end{longtable}
				
				
		\begin{longtable}{rcl}
			\hline
				$ (\!\textit{Expr})\ E $ & $ ::= $ & $ v \mid val \mid ... $ \\
				
				
				
				$ (\!\textit{Stmt})\ C $ & ::= & $ \textbf{skip} \mid \textbf{stop} \mid x \assign E \mid \textbf{send}(E_{\textit{tid}}, E_{\textit{Dest}}, E_{mc})  $ \\
				
				&& $ \mid v_m \assign \textbf{recv}()  \mid x \assign f(E) \mid  \textbf{return } E \mid \textbf{noret}  $ \\
				&& $\mid \textbf{if}(E)\ C \textbf{ else } C  \mid C;C \mid \textbf{while}(E) C  $ \\
				$ (\!\textit{Library})\ L $ & ::= & $ \{ f_1 \rightsquigarrow (x_1, C_1), ..., f_n \rightsquigarrow (x_n, C_n) \} $\\

				$ (\!\textit{M-\!Volcab})\ \mathcal{M} $ & $ \in $ & $ \mathscr{P}(\!\textit{MsgCont}) $ \\

				$ (\!\textit{Protocol})\ P $ & ::= & $ (\mathcal{M}, C_{\textit{serv}}, L) $ \\
				
				$ (\!\textit{Prog})\  {W} $ & ::= & $ \textbf{let } P \textbf{ in } C \parallel ... \parallel C  $ \\
				
				\hline
				
				\multicolumn{3}{c}{\textbf{Figure 2.} Syntax of the Programming Language} \\
			
		\end{longtable}
	
	
	Here are some notes about figure 1 and 2:
	\begin{itemize}
		\item In this document, $ \textit{Nat} $ is the set of natural numbers, i.e. \{ 0,1,2,....\}; $ \textit{Nat}^{+} $ is the set of all positive natural numbers, i.e. $ \textit{Nat} \setminus \{0\} $; $ \textit{Int} $ is the set of all integers; $ [n] $ is an abbriviation for some subset of $ \textit{Nat}^{+} $, e.g. $ [n] = \{1,2,..., n\} $


		\item 对协议$ P $, $ \mathcal{M} $是协议的message volcabulary, $ L $是供client调用的库函数集,$ C_{\textit{serv}} $是协议提供的服务代码,是non-faulty replica型thread的初始程序
		
		\item 引入节点的物理地址是为了以建模"运行在物理地址为$ x $的节点可以发送线程$ y $签名的消息"这件事。为了简化,本文档对world configuration初始化时定义运行在节点上的线程id以及对应节点的物理地址是同一个值,即1号节点运行1号线程,2号节点运行2号线程......
		
		二者进行区分主要体现在\textbf{send}, \textbf{recv}语句上。发送消息时,线程将物理地址作为目的地,不关心运行在该物理地址上的线程id是什么。但在根据消息做判断时(例如想知道将目前已收到的的消息是否来自一个quorum)只关心线程id,不关心该线程运行在那个节点上。物理地址仅起到一个标记消息源的作用,方便对方根据地址进行reply。
		
		\item \textbf{send}的语义由标准单播改为标准多播(multicast, 便于表达atomically broadcast). 其中$ E_\textit{tid} $是线程id表达式,$ E_\textit{Dest} $是集合表达式, 计算后应为目的节点物理地址的集合。$ \textbf{send} $语句是说:将用$ tid $私钥签名的消息发送到由$ \textit{Dest} $物理地址集合标记的节点上去,消息内容为$ mc $
		\item $ \textbf{recv} $语句是说:接收一个物理地址集\ 中包含自己所在节点地址的消息,并将源物理地址$ pa $,声称发送此消息的线程的id $ tid $,消息的内容$ mc $ 3者存储到$ v_m $变量中
		
		\item \textbf{stop}语句将出现在每个初始faulty client程序的末尾,用于在faulty client终止前执行时产生$ (t, \texttt{stop}) $事件(详见Initial World Configuration定义)
		
	\begin{comment}
		\item 下面定义一些全局函数
			\begin{itemize}
				\item 物理地址映射函数$ \ulcorner t \urcorner $,其中$ t $是线程id,$ \ulcorner t \urcorner $是运行线程$ t $的节点的物理地址,不同线程运行在不同节点上,且不可通过物理地址推算线程id
				\begin{kfomula}{0.2}
					\forall t_1, t_2.\ \ulcorner t_1 \urcorner = \ulcorner t_2 \urcorner \implies t_1 = t_2\\
					\red{\forall pa.\ \ulcorner pa \urcorner^{-1} \textit{ undefined }}
				\end{kfomula}
			
%				\item $ \msign{m}{t} $返回消息$ m $由线程$ t $的私钥加密的结果
				
				\item 理想message digest 函数$ D: \textit{Msg} \rightharpoonup \textit{Nat} $满足无冲撞性和不可求逆性
					\begin{kfomula}{0.1}
						\forall m, m'.\ D(m) = D(m') \implies m = m' \\
						\red{\forall d.\ D^{-1}(d) \textit{ undefined }}
					\end{kfomula}
					
%					$  $
			\end{itemize}
		
			\end{comment}
		
		
%		下面是文档中出现的special modifiable local variable
%		\begin{itemize}
%			\item set variable $ \inbuf $ : 是消息接收buffer,初始时就定义在每个线程的$ \sigma_o $中,\textbf{recv}操作将直接把接收的消息添加到此集合中
			
			
			
%		\end{itemize}
		
	\end{itemize}
	
	
	\subsection{States and Event Trace}
	
	
		\begin{longtable}{rcl}
			\hline			
			
			$ (\!\textit{Mem})\ \sigma $ & $ \in $ & $ \textit{PVar} \rightharpoonup \textit{Val} $\\
			
			
			$ (\!\textit{CallStk})\ \kappa $ & ::= & $ \circ \mid  (\sigma_l, x, C)$\\
			
			$ (\!\textit{LcState}) \ \delta $ & ::= & $ (\sigma_c, \sigma_o, \kappa)  $\\
			
			$ (\!\textit{CltStates})\ \varDelta  $ & ::= & $ \{ t_1 \rightsquigarrow {\delta}_1 , ..., t_n \rightsquigarrow {\delta}_n \} $\\
			
			$ (\!\textit{Msg})\ m $ & $ = $ & $ \msign{ pa, mc }{t}  $\\
			
			$ (\!\textit{MsgPool})\ {M} $ & $ \in $ & $ \mathscr{P}(\! \textit{AddrSet}  \times \!\textit{AddrSet}\times \! \textit{Msg}) $\\
			
			$ (\!\textit{PState})\ {S} $ & ::= & $ (\varDelta, {M}) $\\			
			\hline
			
			\multicolumn{3}{c}{\textbf{Figure 3.} States} 
		\end{longtable}


		there are three different kinds of event:
		\begin{kfomula}{0.2}
			(\!\textit{Evt})\ e \ \ ::=\ \  (t, f, val) \mid (t, \texttt{ok}, val) \mid (t, \texttt{stop})
		\end{kfomula} 
	
		note that our history can be infinite, thus 
		\begin{kfomula}{0.1}
			(\!\textit{History})\ H \ \ ::=\  \ \epsilon \mid e :: H \qquad\qquad  (\text{co-inductive})
		\end{kfomula} 
	
	\kspace
	
	Here are some notes for {figure 3}:
	\begin{itemize}
		\item {Message Pool} $ M $ is the finite set of tuple $ (\addrSet_\textit{ori}, \addrSet_\textit{cur}, m) $. 
		\begin{itemize}
			\item $ \addrSet_\textit{ori} $是目标节点物理地址的id的集合,由\textbf{send}和\textbf{recv}的语义可知,$ \addrSet_\textit{ori} $一旦在\textbf{send}时确定就不会再改动,目的是保留该消息的目的信息,可用于书写msg-duplication行为的语义。
			
			\item $ \addrSet_{\textit{cur}} $中保存了当前尚未接收$ mc $的线程id的集合。系统通过将目标线程的id $ t $移出$ \addrSet_{\textit{cur}} $标记$ t $已接收该消息(见\textbf{recv}语义)。最终该元组可能变为$ (\addrSet, \emptyset, mc) $, 这种$ \addrSet_{cur} = \emptyset $的消息不会被移出$ M $, 使得在消息发出后的任何时候都可被复制,由此可模拟replay attack. 
			
			\item \red{$ m = \msign{pa, mc}{i}$意为\ 将消息源物理地址$ pa $和消息体$ mc $ 用线程$ i $的私钥进行加密后的东西, 我不是很清楚能不能这样表达}
		\end{itemize}
		
		
		
		\item 下面是文档中出现的special read only local variable
		\begin{itemize}
			\item positive nat variable $ \textit{tid} $: 线程自己的id
						
			\item bool variable $ \textit{faulty} $: 标记自己是不是faulty的
			
			\item set variable $ R_{tid} $记录系统中所有replica的线程id
			
			\item set variable $ R_{pa} $记录系统中所有replica的物理地址
			
			
			\item $ \textit{role} $ 记录线程的角色信息
			
			
			\item set variable $ \textit{Keys} $ : 是该线程持有的用于持有的私钥的集合,correct thread只持有自己的私钥,faulty thread持有所有其他faulty thread的私钥
		\end{itemize}
		
		
	\end{itemize}

	

	
	
	
	\kspace
	
	
	\begin{definition}[\textbf{Well-formed World Configuration}]\ world configuration $ (W, (\varDelta, M)) $ is well-formed \textit{w.r.t.} protocol $ P $, correct client set $ \clientSet $, faulty client set $ \faultyClients $, correct replica set $ \replicaSet $ and faulty replica set $ \faultyReplica $ iff
		
		
		$ \textsf{well-formed}_{P, \clientSet, \faultyClients, \replicaSet, \faultyReplica}(W, (\varDelta, M)) \ \triangleq \  $
		
		
		\qquad $ \begin{array}{ll}
			\exists n.\ \exists i_1, ..., i_n.\ \exists C_{i_1}, ..., C_{i_n}.\ W = \letin{P}{C_{i_1} \parallel ... \parallel C_{i_n}}\\
			
			\land\ (\clientSet \uplus \faultyClients \uplus \replicaSet \uplus \faultyReplica = \{ i_1, ..., i_n \}) \land (|\replicaSet| \geq 3) \land (|\replicaSet| \geq 2\cdot|\faultyReplica| +1 ) \\
			
			
			\land\  (\forall t \in \clientSet \uplus \faultyClients \uplus \replicaSet \uplus \faultyReplica.\ (\varDelta(t).\sigma_o)(\textit{tid}) = t \land (\varDelta(t).\sigma_o)(\textit{R}_{pa}) = \replicaSet \uplus \faultyReplica \land (\varDelta(t).\sigma_o)(\textit{R}_{tid}) = \replicaSet \uplus \faultyReplica) \\
			
%			\land (\inbuf \in \dom{\varDelta(t).\sigma_o})\\
			
			\land\  (\forall t \in \clientSet \uplus \replicaSet.\ (\varDelta(t).\sigma_o)(\textit{faulty}) = \textbf{false} ) \land (\forall t \in \faultyClients \uplus \faultyReplica.\ (\varDelta(t).\sigma_o)(\textit{faulty}) = \textbf{true} )\\
			
			\land\  (\forall t \in \clientSet \uplus \faultyClients.\ (\varDelta(t).\sigma_o)(\textit{role}) = \texttt{client}) \land (\forall t \in \replicaSet \uplus \faultyReplica.\ (\varDelta(t).\sigma_o)(\textit{role}) = \texttt{replica} )\\
			
			\land\ (\forall t \in \clientSet \uplus \replicaSet.\ \varDelta(t).\sigma_o(\textit{Keys}) =  \{ t \})  \\
			
			\land\ (\forall t \in \faultyClients .\ \varDelta(t).\sigma_o(\textit{Keys}) =  \faultyClients) \land (\forall t \in \faultyReplica.\ \varDelta(t).\sigma_o(\textit{Keys}) =  \faultyReplica) \\
			
%			\land\ (\forall t \in \faultyClients \uplus \faultyReplica.\ (\varDelta(t).\sigma_o)(F_r) = \faultyReplica \land (\varDelta(t).\sigma_o)(F_c) = \faultyClients)\\
			
		\end{array} $
	\end{definition}

	\textsf{well-formed}意味着
	\begin{itemize}
		\item $ \clientSet, \faultyClients, \replicaSet, \faultyReplica $互无交集,且correcr replica的数量不小于$ 3 $且不小于faulty-replica的两倍加1(quorum). 这个来自协议的假设是较为general的, 我看过的byzantine-tolerant protocol对replica的数量都有这个要求。
		
%		且replica线程的数量必须为$ 3k+1 $,其中$ k $是某个正整数,故障的replica线程的个数不超过$ k $. 实际上replica的线程数目可以任意,只要大于故障线程的3倍即可。这里$ 3k+1 $应该是取最弱的条件,只要能证明$ 3k+1 $下byznearizability成立,增加correct replica线程的数目byznearizability仍然能成立
		
		\item 每个线程知道自己的线程id和物理地址
		
		\item 每个线程知道系统中那些线程是replica即(replica的线程id), 以及他们的物理地址。
		
		\item 每个线程知道自己有无故障, 知道自己是client还是replica
		
		\item 故障replica有系统中所有故障replica的private key\\
			故障client有系统中所有故障client的private key \\
			(用于faulty threads之间的协作, 建模规定faulty replica只能使用faulty replica的签名,faulty client只能使用faulty client的签名。现实场景中可能存在faulty client"扮演"faulty replica的情况,这样做只是相当于被扮演的faulty replica变多线程了,提高其效率,并不会打破faulty replica数目限制。也可能存在faulty replica扮演faulty client的情况,对这种我们可以通过在world configuration中新增一个faulty client模拟之)
		
	\end{itemize}

	
	
	
	\subsection{Operational Semantics}
	
	
	
	
	\begin{comment}
	$ \textsf{stop-faultyClt}(\sigma, t) \ \triangleq\ \left\{
	\begin{array}{ll}
		(t, \texttt{stop}) & \sigma(\textit{faulty}) = \textbf{true} \land  \sigma_o(\textit{role}) = \texttt{client}\\
		\tau & \textit{o.w.} \\
	\end{array}
	\right. $
	\end{comment}
	
	\subsubsection{Global Transition Rules}
	
	\vspace{0.2cm}
	
	\begin{center}
		$
		\inferrule[Glb-Int]
		{ (C_t, \varDelta(t), M) \longcltstep{\ e\ }{t,P} (C_t', \delta', M') }
		{ (\textbf{let } P \textbf{ in } C_1 \parallel ... \ C_t\ ... \parallel C_n,\ (\varDelta, M) ) \longworldstep{\ e\ } (\textbf{let } P \textbf{ in } C_1 \parallel ...\ C_t' \ ... \parallel C_n,\ (\varDelta\{t \rightsquigarrow \delta'\}, M')) }
		$
		
		\kspace
		
		$
		\inferrule[Glb-Met-Misbehave]
		{ M \xrsquigarrow{ \ \ \ } M' }
		{ (\textbf{let } P \textbf{ in } C_1 \parallel ... \parallel C_n,\ (\varDelta, M) ) \longworldstep{\ \ \ } (\textbf{let } P \textbf{ in } C_1 \parallel ... \parallel C_n,\ (\varDelta, M')) }
		$
		
		\kspace
		
		$
		\inferrule[Glb-Thd-Crash]
		{ C_t \neq \textbf{skip} \\ \varDelta(t).\sigma_o(\textit{faulty}) = \textbf{true} }
		{ (\textbf{let } P \textbf{ in } C_1 \parallel ... \ C_t\ ... \parallel C_n,\ (\varDelta, M) ) \longworldstep{ \ \ \ } (\textbf{let } P \textbf{ in } C_1 \parallel ...\ \textbf{stop} \ ... \parallel C_n,\ (\varDelta, M)) }
		$
			
	\end{center}

	

%	\begin{itemize}
%		\item crash行为定义成修改程序为skip
%		\item 如果t是faulty client, 则产生事件$ (t, \texttt{stop}) $, 否则silent. 因为在描述byznearizability时只需要faulty client的stop标记
%	\end{itemize}
		
	\subsubsection{Network Misbehave Rules}
	
	\vspace{0.2cm}
	
	\begin{center}
		$ \inferrule[Msg-Lose]
		{ (\addrSet_\textit{ori}, \addrSet_\textit{cur}, m) \in M \\ pa \in \addrSet_\textit{cur} }
		{M \xrsquigarrow{\ \ \ }  (M \setminus (\addrSet_\textit{ori}, \addrSet_\textit{cur}, m)) \cup (\addrSet_\textit{ori}, \addrSet_\textit{cur} \setminus \{ pa \}, m)}
		$
		\qquad		
		$ \inferrule[Msg-Dup]
		{ (\addrSet_\textit{ori}, \addrSet_\textit{cur}, m) \in M \\ pa \in \addrSet_\textit{ori}   }
		{M \xrsquigarrow{\ \ \ }   (M \setminus (\addrSet_\textit{ori},\addrSet_\textit{cur}, m)) \cup (\addrSet_\textit{ori}, \addrSet_\textit{cur}\cup\{ pa \}, m) }
		$
	\end{center}
	
	
	
%	从Dup规则可以看到,mc可能原不打算发给$ t $,也可以凭空产生发送给t。这种行为可由网络故障引起。
	
	\subsubsection{Thread Transition Rules}
	\noindent \textbf{Thread transition inside method call:}
	
	\vspace{0.2cm}
	
	\begin{center}
		$ 
		\inferrule[Ret]
		{	\llbracket E \rrbracket_{ \sigma_o \uplus \sigma_l} = {val} }
		{ (\textbf{E}[\textbf{return }E], (\sigma_c, \sigma_o, (\sigma_l, x, C)), M ) \longcltstep{(t, \texttt{ok}, val)}{t, P} (C, (\sigma_c\{ x \rightsquigarrow \textit{val}\}, \sigma_o, \circ),  M ) }
		$
%		

		\kspace
		
		$
		\inferrule[Assn]
		{ \sigma = \sigma_o \uplus \sigma_l \\ \llbracket E \rrbracket_{ \sigma } = val \\ \sigma_o' \uplus \sigma_l' = \sigma\{ x \rightsquigarrow val \} \\ \textsf{dom}(\sigma_l) = \textsf{dom}(\sigma_l') }
		{ (\textbf{E}[x \assign E], (\sigma_c, \sigma_o, (\sigma_l, x, C)),  M) {\longcltstep{ \ \ \  }{t, P}\ (\textbf{E}[\textbf{skip}], (\sigma_c, \sigma_o', (\sigma_l', x, C)),  M  ) }}
		$
		
		
		\kspace
		
		$ \inferrule[Stop-Clt]
		{ \sigma_o(\textit{faulty}) = \textbf{true}  \\ \sigma_o(\textit{role}) = \texttt{client} }
		{(\textbf{E}[\textbf{stop}], (\sigma_c, \sigma_o, \kappa), M) \longcltstep{(t, \texttt{stop}) }{t,P} (\textbf{skip}, (\sigma_c, \sigma_o, \kappa), M) } 
		$
		\ 
		$ \inferrule[Stop-Rpc]
		{ \sigma_o(\textit{faulty}) = \textbf{true}  \\ \sigma_o(\textit{role}) = \texttt{replica} }
		{(\textbf{E}[\textbf{stop}], (\sigma_c, \sigma_o, \kappa), M) \longcltstep{\ \ \ }{t,P} (\textbf{skip}, (\sigma_c, \sigma_o, \kappa), M) } 
		$
		
		
		\begin{comment}
		
		\kspace
		
		$ \inferrule[Send]
		{ \sigma = \sigma_o \uplus \sigma_l \\   to = \eevalu{E_{to}}{\sigma} \\ mc = \eevalu{E_{mc}}{\sigma} \\ mc \in P.\mathcal{M} \\  }
		%
		{(\textbf{E}[\textbf{send}(E_{to}, E_{mc})], (\sigma_c, \sigma_o, (\sigma_l, x, C)),  M) \longcltstep{ \ \ \ }{t, P} (\textbf{E}[\textbf{skip}], (\sigma_c, \sigma_o, (\sigma_l, x, C)),  M \uplus \{ (t, to, \textbf{false}, mc) \} ) } $
		\end{comment}
		\kspace
		
		
		$ \inferrule[Send]
		{ \sigma = \sigma_o \uplus \sigma_l \\ \textit{tid} = \eevalu{E_\textit{tid}}{\sigma} \\ \textit{tid} \in \sigma_o(\textit{Keys}) \\
		%
		\mathcal{A} = \eevalu{E_{\textit{Dest}}}{\sigma} \\ mc = \eevalu{E_{mc}}{\sigma} \\ mc \in P.\mathcal{M}  }
		%
		{(\textbf{E}[\textbf{send}(E_\textit{tid}, E_{\textit{Dest}}, E_{mc})], (\sigma_c, \sigma_o, (\sigma_l, x, C)),  M) \longcltstep{ \ \ \ }{t, P} (\textbf{E}[\textbf{skip}], (\sigma_c, \sigma_o, (\sigma_l, x, C)),  M \cup \{ (\mathcal{A}, \addrSet, \msign{ t, mc}{\textit{tid}}) \} ) } $
		
		
		\kspace
		
		
		$ \inferrule[Recv]
		{
			\sigma = \sigma_l \uplus \sigma_o  \\  \\ (\addrSet_\textit{ori}, \addrSet_\textit{cur}, m) \in M \\ m = \msign{pa, mc}{\textit{tid}} \\\\
			t \in \addrSet_\textit{cur} \\ \sigma_o'\uplus \sigma_l' = \sigma\{ v_m \rightsquigarrow (pa, \textit{tid}, mc) \} \\ \dom{\sigma_l} = \dom{\sigma_l'}\\\\
			M' = (M \setminus \{ (\addrSet_\textit{ori}, \addrSet_\textit{cur}, m) \}) \cup \{ (\addrSet_\textit{ori}, \addrSet_\textit{cur} \setminus \{t\}, m) \}
		}
		{ (\textbf{E}[v_m \assign \textbf{recv}()], (\sigma_c, \sigma_o, (\sigma_l, x, C)), M) \longcltstep{ \ \ \ }{t, P} (\textbf{E}[\textbf{skip}], (\sigma_c, \sigma_o', (\sigma_l', x, C)), M') } $
		
		\kspace
		
		$ \inferrule[WhileCont]
		{ \llbracket E \rrbracket_{\sigma_o \uplus \sigma_l} = \textbf{true} }
		{(\textbf{E}[\textbf{while}(E)C], (\sigma_c, \sigma_o, (\sigma_l, x, C')), M) \longcltstep{\ \ \ }{t,P} (\textbf{E}[C;\textbf{while}(E)C], (\sigma_c, \sigma_o, (\sigma_l, x, C')), M) }
		$
		
		\kspace
		
		$ \inferrule[WhileEnd]
		{ \llbracket E \rrbracket_{\sigma_o \uplus \sigma_l} = \textbf{false} }
		{(\textbf{E}[\textbf{while}(E)C], (\sigma_c, \sigma_o, (\sigma_l,x, C')), M) \longcltstep{\ \ \ }{t,P} (\textbf{E}[\textbf{skip}], (\sigma_c, \sigma_o, (\sigma_l,x, C')), M) }
		$	
		
		\kspace
		
		$ \inferrule[If1]
		{ \llbracket E \rrbracket_{\sigma_o \uplus \sigma_l} = \textbf{true} }
		{(\textbf{E}[\textbf{if}(E)C_1\textbf{ else } C_2], (\sigma_c, \sigma_o, (\sigma_l,x, C)), M) \longcltstep{\ \ \ }{t,P} (\textbf{E}[C_1], (\sigma_c, \sigma_o, (\sigma_l,x, C)), M) } 
		$
		
		\kspace
		
		$ \inferrule[If2]
		{\llbracket E \rrbracket_{\sigma_o \uplus \sigma_l} = \textbf{false} }
		{(\textbf{E}[\textbf{if}(E)C_1\textbf{ else } C_2], (\sigma_c, \sigma_o, (\sigma_l,x, C)), M) \longcltstep{\ \ \ }{t,P} (\textbf{E}[C_2], (\sigma_c, \sigma_o, (\sigma_l,x, C)), M) } 
		$
	
		
		\end{center}
		
		\vspace{0.2cm}
		
		\noindent \textbf{Thread transition outside method call:}
		
		\vspace{0.2cm}
		
		\begin{center}
		
%		\begin{comment}
		$
		\inferrule[Call]
		{ P.L(f) = (y, C) \\ \llbracket E \rrbracket_{\sigma_c} = \textit{val} \\
		x \in \textsf{dom}(\sigma_c) }
		%
		{ (\textbf{E}[x\assign f(E)], (\sigma_c, \sigma_o, \circ), M) \longcltstep{(t,f,\textit{val})}{t, P} (C;\textbf{noret}, (\sigma_c, \sigma_o, (\{ y \rightsquigarrow \textit{val}\}, x, \textbf{E}[\textbf{skip}])), M) }
		$

		\kspace
		
		$
		\inferrule[Clt-Assn]
		{ \llbracket E \rrbracket_{ \sigma_c } = val }
		{ (\textbf{E}[x \assign E], (\sigma_c, \sigma_o, \circ), M) {\longcltstep{ \ \ \  }{t, P}\ (\textbf{E}[\textbf{skip}], (\sigma_c\{  x \rightsquigarrow val \}, \sigma_o, \circ), M) }}
		$
		
		\kspace
		
		$ \inferrule[Clt-If1]
		{ \llbracket E \rrbracket_{\sigma_c} = \textbf{true} }
		{(\textbf{E}[\textbf{if}(E)C_1\textbf{ else } C_2], (\sigma_c, \sigma_o, \circ), M) \longcltstep{\ \ \ }{t,P} (\textbf{E}[C_1], (\sigma_c, \sigma_o, \circ), M) } 
		$
		
		
		\kspace
		
		$ \inferrule[Clt-If2]
		{\llbracket E \rrbracket_{\sigma_c} = \textbf{false} }
		{(\textbf{E}[\textbf{if}(E)C_1\textbf{ else } C_2], (\sigma_c, \sigma_o, \circ), M) \longcltstep{\ \ \ }{t,P} (\textbf{E}[C_2], (\sigma_c, \sigma_o, \circ), M) } 
		$
		
		\kspace
		
		$ \inferrule[Clt-WhileCont]
		{\llbracket E \rrbracket_{\sigma_c} = \textbf{true} }
		{(\textbf{E}[\textbf{while}(E)C], (\sigma_c, \sigma_o, \circ), M) \longcltstep{\ \ \ }{t,P} (\textbf{E}[C;\textbf{while}(E)C], (\sigma_c, \sigma_o, \circ), M) } 
		$
		
		\kspace
		
		$ \inferrule[Clt-WhileEnd]
		{\llbracket E \rrbracket_{\sigma_c} = \textbf{false} }
		{(\textbf{E}[\textbf{while}(E)C], (\sigma_c, \sigma_o, \circ), M) \longcltstep{\ \ \ }{t,P} (\textbf{E}[\textbf{skip}], (\sigma_c, \sigma_o, \circ), M) } 
		$
		
		
%		\end{comment}	
	\end{center}
	
	
	
	
	\kspace
	
	\begin{comment}
	\subsubsection{Multiple-Steps of Global Transition}
	
	%	we abbreviate $ (\exists H.\ (W,S) \longmworldstep{\ H \ } (W',S)) $ as $ (W,S) \longmworldstep{\ \ \ } (W',S) $
	
	$ (W,S) \longmworldstep{\ H\ } (W',S') \ \triangleq\ \exists k.\ (W,S) \longkworldstep{\ H \ }{k} (W', S') $
	
	\begin{center}	
		$
		\inferrule[Wzero-Step]
		{\ }
		{(W,S) \longkworldstep{\ \epsilon \ }{0} (W, S)}
		$
		\qquad 
		$
		\inferrule[W-Steps]
		{(W,S) \longworldstep{\ e \ } (W'',S'') \\
			(W'', S'') \longkworldstep{\ H' \ }{k} (W',S') }
		{(W,S) \longkworldstep{\ e::H' \ }{k\!\!+\!\!1} (W', S')}
		$
		
	
		
	\end{center}
\end{comment}	
	\pagebreak

	
	\section{Byznearizablity}
		
	In this section we formalize correctness condition byznearizablity for distributed protocols.
	
	
%	x \assign \textbf{faulty-call}(f, E_\textit{val}, E_\textit{tid})
	
	\subsection{Protocol Specification and Abstract Machine}
	
	\begin{longtable}{rcl}
		\hline
		
		$ (\!\textit{HistOper})\ o $ & ::= & $ (t, f, val)::(t, \texttt{ok}, val)::\epsilon $ \\
		
		$ (\!\textit{MemBlocks}) \ B $ & ::= & $ \{t_1 \rightsquigarrow \sigma_1, ..., t_n \rightsquigarrow \sigma_n \} $ \\
		
		$(\!\textit{AbsObj})\ \theta$ & $ \in $ & $ \textit{PVar} \rightharpoonup \textit{AbsVal} $\\
		
		$ (\!\textit{MSpec})\ \gamma  $ & $ \in $ & $ \textit{Val} \rightharpoonup \textit{AbsObj} \rightharpoonup \textit{Val} \times \textit{AbsObj} $ \\
		
		$ (\!\textit{PSpec}) \ \Gamma $ & ::= & $ \{ f_1 \rightsquigarrow \gamma_1, ..., f_m \rightsquigarrow \gamma_m \} $ \\
		
		$ (\!\textit{AbsStmt})\ \mathbb{C} $& ::= &  $ C \mid \textbf{fexec}(f, val, \textit{tid}) \mid \textbf{fret}(val) $ \\
		&&  $ \mid  \textbf{aftermath}(H) $ \\
		
		$ (\!\textit{AbsProg})\ \ \mathbb{W} $ & ::= & $ \textbf{with } \Gamma \textbf{ do } \mathbb{C} \parallel ... \parallel \mathbb{C} $ \\
		
		$ (\!\textit{AbsStk})\ \textit{ak} $ & ::= & $ (x, \mathbb{C}) \mid \circ $\\
		
		$ (\!\textit{AbsLocState})\ \textit{al} $ & ::= & $ (\sigma_c, \theta, ak) $\\
		
		$ (\!\textit{AbsPool})\ \mathbb{K} $ & ::= & $ \{ t_1 \rightsquigarrow ak_1, ..., t_n \rightsquigarrow ak_n \} $\\
		
		$ (\!\textit{AbsState})\ \mathbb{S}  $& ::= & $ (B_c, \theta, \mathbb{K}) $\\
		\hline
	\end{longtable}
	

	\subsection{Abstract Operational Semantics}
	
%	\kcomment{白茫茫大地真干净!}
	
	\begin{center}	
		
		$ \inferrule[AGlb-Int]
		{  (\mathbb{C}_t, (B_c(t), \theta, \mathbb{K}(t))) \longbulletarrow{\ H \ }_{t, \Gamma} (\mathbb{C}_t', (\sigma_c', \theta', ak') )  }
		%
		{ (\textbf{with }\Gamma \textbf{ do } \mathbb{C}_1 \parallel ...\mathbb{C}_t ... \parallel \mathbb{C}_n, (B_c, \theta, \mathbb{K})) \longcircarrow{\ H\ } (\textbf{with }\Gamma \textbf{ do } \mathbb{C}_1 \parallel ...\ \mathbb{C}_t'\ ... \parallel \mathbb{C}_n, (B_c\{t \rightsquigarrow \sigma_c'\}, \theta', \mathbb{K}\{ t \rightsquigarrow ak' \}) ) }
		$
		
		\kspace
		
		$
		\inferrule[AGlb-Crash]
		{ \not\exists H.\ \mathbb{C}_t = \textbf{aftermath}(H) \\ \mathbb{C}_t \neq \textbf{skip} \\ B_c(t)(\textit{faulty}) = \textbf{true} \\ o1, ..., o_m \in \textit{HistOper} }
		{ (\textbf{with } \Gamma \textbf{ do } \mathbb{C}_1 \parallel ... \ \mathbb{C}_t\ ... \parallel \mathbb{C}_n,\ (B_c, \theta, \mathbb{K})) \longcircarrow{(t, \texttt{stop})::\epsilon} (\textbf{with } \Gamma \textbf{ do } \mathbb{C}_1 \parallel ...\ \textbf{aftermath}(o_1::...::o_m) \ ... \parallel \mathbb{C}_n,\ (B_c, \theta, \mathbb{K})) }
		$
		
		
		\kspace
		
		$ \inferrule[ACall]
		{ f \in \textsf{dom}(\Gamma) \\
			\llbracket E \rrbracket_{ \sigma_c } = val \\
			x \in \textsf{dom}(\sigma_c)  \\  i \in  \sigma_c(\textit{Keys}) }
		{ (\textbf{E}[x\assign f(E)] , (\sigma_c, \theta, \circ)) \longbulletarrow{\ \epsilon \ }_{t, \Gamma} (\textbf{fexec}(f, val, i), (\sigma_c, \theta, (x, \textbf{E}[\textbf{skip}]) )) } $
		
		
		\kspace
		
		\begin{comment}
		$ \inferrule[AFaulty-Call]
		{	\sigma_c(\textit{faulty}) = \textbf{true} \\ 
			f \in \textsf{dom}(\Gamma) \\
			\llbracket E_\textit{val} \rrbracket_{ \sigma_c } = val \\
			\llbracket E_\textit{tid} \rrbracket_{ \sigma_c } = \textit{tid} \\
			\textit{tid} \in \sigma_c(\textit{Keys}) \\
			x \in \textsf{dom}(\sigma_c) \\  }
		{ (\textbf{E}[x\assign \textbf{faulty-call}(f, E_\textit{val}, E_\textit{tid})] , (\sigma_c, \theta, \circ)) \longbulletarrow{\ \epsilon \ }_{t, \Gamma} (\textbf{fexec}(f, val, \textit{tid}), (\sigma_c, \theta, (x, \textbf{E}[\textbf{skip}]) )) } $
		
		\kspace
		
		\end{comment}
		
		
		$ \inferrule[AExec]
		{  	\Gamma(f)(val)(\theta) = (val', \theta') }
		{(\textbf{fexec}(f, val, \textit{tid}) , (\sigma_c, \theta, ak)) \longbulletarrow{ (\textit{tid}, f, val)::(\textit{tid}, \texttt{ok}, val')::\epsilon }_{ t, \Gamma} (\textbf{fret}(val'), (\sigma_c, \theta', ak))  } $
		
		
		\kspace
		
		$ \inferrule[AFret]
		{ \ }
		{(\textbf{fret}(val') , (\sigma_c, \theta, (x, {C}))) \longbulletarrow{\ \epsilon \ }_{t, \Gamma} ({C}, (\sigma_c\{ x \rightsquigarrow val' \},\theta, \circ))  } $
		\qquad
		$ \inferrule[AAssn]
		{ \eevalu{E}{\sigma_c} = val }
		{(\textbf{E}[x \assign E], (\sigma_c, \theta, \circ)) \longbulletarrow{\ \epsilon \ }_{t,\Gamma} (\textbf{E}[\textbf{skip}], (\sigma_c\{ x \rightsquigarrow  val \}, \theta, \circ)) } 
		$
		
		\kspace
		
		$ \inferrule[AStop]
		{ \  \sigma_c(\textit{faulty}) = \textbf{true} \\ o_1, ..., o_m \in \textit{HistOper}}
		{(\textbf{E}[\textbf{stop}], (\sigma_c, \theta, \circ)) \longbulletarrow{(t, \texttt{stop})::\epsilon}_{t,\Gamma} (\textbf{aftermath}(o_1::...::o_m), (\sigma_c, \theta, \circ)) } 
		$
		
		
		\kspace
		
		$ \inferrule[AAftermath]
		{ \ o = (t, f, val)::(t, \texttt{ok}, val')::\epsilon \\ t \in \sigma_c(\textit{Keys}) \\ f \in \dom{\Gamma} \\ \Gamma(f)(val)(\theta) = (val', \theta')}
		{(\textbf{aftermath}(o::H'), (\sigma_c, \theta, \circ)) \longbulletarrow{\ o\ }_{t,\Gamma} (\textbf{aftermath}(H'), (\sigma_c, \theta', \circ)) } 
		$
		
		
		\kspace
		
			
		$ \inferrule[AIf1]
		{ \llbracket E \rrbracket_{\sigma_c} = \textbf{true} }
		{(\textbf{E}[\textbf{if}(E) \mathbb{C}_1\textbf{ else } \mathbb{C}_2], (\sigma_c, \theta, \circ)) \longbulletarrow{\ \epsilon \ }_{t, \Gamma} (\textbf{E}[\mathbb{C}_1], (\sigma_c, \theta, \circ)) } 
		$\qquad
		$ \inferrule[AIf2]
		{ \llbracket E \rrbracket_{\sigma_c} = \textbf{false} }
		{(\textbf{E}[\textbf{if}(E) \mathbb{C}_1\textbf{ else } \mathbb{C}_2], (\sigma_c, \theta, \circ)) \longbulletarrow{\ \epsilon \ }_{t, \Gamma} (\textbf{E}[\mathbb{C}_2], (\sigma_c, \theta, \circ)) } 
		$
		
		\kspace
		
		$ \inferrule[AWhileCont]
		{ \llbracket E \rrbracket_{\sigma_c} = \textbf{true} }
		{(\textbf{E}[\textbf{while}(E)\mathbb{C}], (\sigma_c, \theta, \circ)) \longbulletarrow{\ \epsilon \ }_{t,\Gamma} (\textbf{E}[\mathbb{C};\textbf{while}(E)\mathbb{C}], (\sigma_c, \theta, \circ)) } 
		$\qquad
		$ \inferrule[AWhileEnd]
		{ \llbracket E \rrbracket_{\sigma_c} = \textbf{false} }
		{(\textbf{E}[\textbf{while}(E) \mathbb{C}], (\sigma_c, \theta, \circ)) \longbulletarrow{\ \epsilon \ }_{t,\Gamma} (\textbf{E}[\textbf{skip}], (\sigma_c, \theta, \circ)) } 
		$
		
		
		
	% \end{center}
	
%	注意这里\ifrule{AStop} 前条件要求$ ak $为空,与\ifrule{Stop}中$ \kappa \neq \circ $不一样
	
	\begin{comment}
		
	
	\subsection{Multiple-Steps of Abstract Global Transition}
	
	\vspace{0.2cm}
	\begin{center}
		$
		\inferrule[AWzero-Step]
		{\ }
		{(\mathbb{W},\mathbb{S}) \longmcircarrow{\ \epsilon\ } (\mathbb{W}, \mathbb{S}) }
		$
		\qquad
		$
		\inferrule[AWe-Steps]
		{(\mathbb{W}, \mathbb{S}) \longcircarrow{\ H_1 \ } (\mathbb{W}'', \mathbb{S}'') \\
			(\mathbb{W}'', \mathbb{S}'') \longmcircarrow{\ H_2\ } (\mathbb{W}', \mathbb{S}') }
		{(\mathbb{W},\mathbb{S}) \longmcircarrow{\ H_1::H_2\ } (\mathbb{W}', \mathbb{S}') }
		$
	\end{center}
	\end{comment}
	
	
	
	\begin{itemize}
		\item 为了建模"扮演",函数调用语句 $ x \assign f(E) $ 是非确定性的。由\textit{Keys}的初始定义(见Initial abstract world configuration定义),correct client只能以自己的名义执行$ f $,但faulty client可以非确定地以某个faulty client的名义执行$ f $
		\item 为了建模"余波",定义$ \textbf{aftermath}(H) $语句
	\end{itemize}
	
	\kspace
	
	\subsection{Byznearizable relation between histories}
	
	the following are some trivial predicates and a function for events: 
	
	
	$$
	\begin{array}{l}
		 \textsf{is-inv}(e)\ \triangleq\ \exists t,f,val.\ e = (t,f,val) \\
		 \textsf{is-res}(e)\ \triangleq\ \exists t , val.\  e = (t,\texttt{ok},val) \\
		 \textsf{is-stop}(e)\ \triangleq\ \exists t.\  e = (t, \texttt{stop}) \\		
	\end{array}
	\qquad
	\textsf{get-tid}(e)\  \triangleq\ \left\{
	\begin{array}{ll}
	t & \textsf{if }e = (t, \wildcard, \wildcard)  \\
	t & \textsf{if }e = (t, \texttt{stop})  \\
	\textit{undefined} & \textit{o.w.}
	\end{array}
	\right.$$
	
	\vspace{0.3cm}
	
	the following are functions for history, note that $ H(i) $ is the i-th event in $ H $, $ H|_t $ is the projection of history $ H $ on thread t, we also call $ H|_t $ a $ \textit{t-component} $ of $ H $. We use $ |H| $ to represent the length of history $ H $.
	%
	$$ H(i)\ \triangleq\ \left\{
	\begin{array}{ll}
	H'(i-1) & \textsf{if } i > 1 \land H = e::H'\\
	e & \textsf{if } i = 1 \land H = e::H'\\
	\textit{undefined} & \textit{o.w.}
	\end{array}
	\right.
	\qquad
	|H| \ \triangleq\ \left\{
	\begin{array}{ll}
		1 + |H'| & \textsf{if } H = e::H' \\
		0 & \textit{o.w.}
	\end{array}
	\right.
	 $$ 
	
	
	$$ H|_t \ \triangleq \ \left\{
	\begin{array}{ll}
		\epsilon & \textsf{if }H ::= \epsilon \\
		e :: H'|_t & \textsf{if }(H ::= e::H') \land (t = \textsf{get-tid}(e))\\
		H'|_t & \textit{o.w.}
	\end{array}
	\right.
	$$
	
	
	\begin{comment}
		
	
	
	\begin{definition}{{A response $ e' $ \textit{matches} an invocation $ e $ in $ H $, denoted $\textsf{match}(e, e') $}, if they have the same thread id:}
		
		$ \textsf{match}(e,e') \triangleq \textsf{is-inv}(e) \land \textsf{is-res}(e') \land (\textsf{get-tid}(e) = \textsf{get-tid}(e')) $
		
	\end{definition}
	
	\begin{definition}{{A History }$ H ${ is }\textit{seqential}{ iff 
		\begin{itemize}
			\item the first event is an invocation or stop
			\item every response ared followed a inv or stop: 
			\item dfg
		\end{itemize}}}
		
		$ \textsf{seq}(H) \triangleq \left\{
		\begin{array}{ll}
		\textbf{true} & \textsf{if } H = \epsilon \\
		\textsf{is-inv}(e) & \textsf{if }	H = e::\epsilon	\\
		\textsf{match}(e, e') \land \textsf{seq}(H') & \textsf{if } H = e::e'::H' 
		\end{array}
		\right. $
		
	\end{definition}
	
	
	\begin{definition}{{A history} $ H $ is \textit{well-formed} iff $ H|_t $ is sequential for all $ t $:}
		
		$ \textsf{well-formed}(H) \triangleq \forall t.\  \textsf{seq}(H|_t)$
	\end{definition}

	The well-formedness formalize two properties of history. First, it ensures all responses have corresponding invocation. Second, the history projection on each thread is "sequential". 
	
	
	\begin{definition}{{An \textit{extensions} of $ H $, denoted \textsf{extensions}$ (H) $, is a set of histories constructed by appending responses to zero or more pending invocations of $ H $}}
		
		$\textsf{extensions} (H) \triangleq \{ H' \mid \exists H_\textit{ok}.\  (H' = H::H_\textit{ok}) \land \textsf{well-formed}(H') \land \forall i.\  \textsf{is-res}(H_\textit{ok}(i))  \} $
	\end{definition}
	
	\begin{definition}{{The \textit{complete} histories of $ H $, denoted $ \textsf{completions}(H)$, is a set of subsequences of $ \textsf{extensions}(H) $ consisting of all matching invocations and responses}: }
		
		$ \textsf{completions}(H) \triangleq \{ \textsf{truncate}(H') \mid H' \in \textsf{extensions}(H) \} $
		
		\vspace{0.2cm}
		
		where \textit{truncate}$ (H) $ is the maximal complete sub-history of $ H $:
		
		$ \textsf{truncate}(H) \triangleq \left\{
		\begin{array}{ll}
		\epsilon & H = \epsilon \\
		e::\textsf{truncate}(H') & \textsf{if } H = e::H' \land (\textsf{is-res}(e) \lor \exists i.\  \textsf{match}(e, H(i))) \\
		\textsf{truncate}(H') & \textsf{if } H = e::H' \land (\lnot (\textsf{is-res}(e) \lor \exists i.\  \textsf{match}(e, H(i)))) \\
		\end{array}
		\right. $
		
	\end{definition}
	
	\begin{comment}
		\begin{definition}{{two histories are }\textit{weakly equivalent} {iff their projection to each thread are equal}:}
		
		$ H \equiv H'\ \triangleq \ \forall t.\  H|_t = H'|_t $
		
		\end{definition}
	\end{comment}
	
	\begin{definition}[\textbf{byznearizable relation}] A history $ H $ is \textit{linearizable} to $ H' $ \textit{w.r.t.} thread set $ \clientSet $ and  $ \faultyClients $, denoted $ H \preceq_\textsf{lin}^{\clientSet, \faultyClients} H' $ iff 


	\quad $ \begin{array}{ll}
		\textsf{byz1. }  (\forall t \in \clientSet.\ H|_t = H'|_t)  \\
		\textsf{byz2. } (\exists \pi.\ (\forall i_1, i_2.\ \pi(i_1) = \pi(i_2) \implies i_1 = i_2) \land (\forall i.\ H(i) = H'(\pi(i)))\\
		 \qquad \qquad  \land \ (\forall i, j.\ (i < j) \land (\textsf{is-res}(H(i)) \lor  \textsf{is-stop}(H(i))) \land (\textsf{is-inv}(H(j)) \lor \textsf{is-stop}(H(j)) ) \implies \pi(i) < \pi(j))\\
		\textsf{byz3. } (\forall t \in \faultyClients.\ (\exists i.\ H(i) = (t,\texttt{stop}))) \implies (\forall t \in \faultyClients.\ \exists n.\ |\{ i \mid \textsf{get-tid}(H'(i)) = t \}| \leq n))  \\
		
%		& \land \ (\forall t \in \faultyClients.\ (t,\texttt{stop}) \in H) \implies (\forall t.\ (\exists H_1', H_2'.\ H' = H_1:: (t, \texttt{stop}) ::H_2) \land (\exists n.\ |\{ i \mid \textsf{get-tid}(H_2(i)) = t \}| \leq n))  \\
		\end{array} $
		
		\vspace{0.2cm}
		
%		$ e \in H \ \triangleq \ \exists H_1, H_2.\ H = H_1::e::H_2 $
		
		
	\end{definition}


	\subsubsection{Byznearizable Protocol}
	
%	A linearizable object is one whose concurrent histories are linearizable w.r.t. some sequential specification. To define linearizability, we first need to define what is \textit{object's concurrent histories}. So we formalize the set of histories generated by the set of programs under following restriction:
%	\begin{itemize}
%		\item all acceptor threads in the program run and only run $ \kacc() $ in the beginning
%		\item all proposer threads could run all possible statements which use function $ \kpropP $ only
%	\end{itemize}
	
	\begin{definition}[\textbf{Produce Concurrent History}]\ history $ H $ is produced by executing $ W $ from state $ S $
		
		\vspace{0.2cm}
		
		$\inferrule[]
		{(W,S) \longworldstep{\ e \ } (W',S') \\ \textsf{Hist}(W',S',H') }
		{\textsf{Hist}(W,S,e::H')}$
		\quad
		$\inferrule[]
		{(W,S) \longworldstep{\ \ \ } (W',S') \\ \textsf{Hist}(W',S',H) }
		{\textsf{Hist}(W,S,H)}$
		
	\end{definition}
	
	\begin{definition}[\textbf{Produce Sequential History}]\ history $ H $ is produced by executing $ \mathbb{W} $ from state $ \mathbb{S} $
		
		\vspace{0.2cm}
		
		$\inferrule[]
		{(\mathbb{W},\mathbb{S}) \longcircarrow{\ H_1 \ } (\mathbb{W}',\mathbb{S}') \\ \textsf{Hist}(\mathbb{W}',\mathbb{S}',H_2) }
		{\textsf{Hist}(\mathbb{W},\mathbb{S},H_1::H_2)}$

	\end{definition}
	
	\vspace{0.2cm}
	
	\begin{definition}[\textbf{Initial World Configuration}]\ 
		
		$ \begin{array}{ll}
			(W, (\varDelta, \emptyset)) = \textsf{pInit}_{\clientSet, \faultyClients, \replicaSet, \faultyReplica}(P, B_o) \ \triangleq \  & \textsf{well-formed}_{P, \clientSet, \faultyClients, \replicaSet, \faultyReplica}(W, (\varDelta, \emptyset)) \\
			
			& \land \ (\forall t \in \replicaSet.\ C_t = P.C_{\textit{serv}} )\\
			
			& \land\  (\forall t \in \replicaSet \uplus \faultyReplica.\ \varDelta(t).\sigma_o = B_o(t))\\
			
			& \land \ (\forall t \in \clientSet.\ \varDelta(t).\kappa = \circ )  \land (\forall t \in \faultyClients \uplus \replicaSet \uplus \faultyReplica.\ \varDelta(t).\kappa = (\emptyset, \wildcard, \textbf{skip}) )  \\
			
%			& \land \ (\forall t \in \clientSet \uplus \faultyClients \uplus \replicaSet \uplus \faultyReplica.\ (\varDelta(t).\sigma_o)(\inbuf) = \emptyset  )
			
			& {\land \ (\forall t \in \faultyClients.\ \exists C.\ C_t = C;\textbf{stop})}
		\end{array} $
	\end{definition}
	
	\textsf{pInit}意味着
	\begin{itemize}
		\item 正确的replica线程初始程序只能是协议的服务程序$ C_\textit{serv} $,故障的(\textit{faulty} = \textbf{true}的) replica线程的初始程序不做任何限制
		
		\item faulty client的初始程序一定是某个程序$ C $与\textbf{stop}语句的串联. 用于标记faulty client的终止
		
		\item $ B_o $用于初始化所有replica线程的初始local object state $ \sigma_o $
		
		\item 所有replica和faulty client线程的local stack 初始都是$ (\emptyset, \wildcard, \textbf{skip}) $. 这是为了符合语义规则,使replica线程能不经函数调用执行\textbf{send}, \textbf{recv}语句({\color{red} 这样定义可能可读性上有些问题})
%		\item 所有线程初始时的接收缓冲池为空(\textit{Inbuf})

	\end{itemize}
	
	
	\begin{definition}[\textbf{Concurrent Histories}]\ 
		 
		$ \histSet{P, B_o, \clientSet, \faultyClients} \ \triangleq\ \left\{ H \ \middle\vert  \begin{array}{ll}
			\exists {W}, {S}, \replicaSet, \faultyReplica.\ ({W}, {S}) = \textsf{pInit}_{\clientSet, \faultyClients, \replicaSet, \faultyReplica}(P, B_o) \ 
			\land \ \textsf{Hist}({W}, {S}, H) 
		\end{array}  \right\} $
	\end{definition}
	
	
	
	\begin{definition}[\textbf{Initial Abstract World Configuration}]\ 
		
		
		$ \begin{array}{ll}
			(\mathbb{W}, (B_c, \theta, \mathbb{K})) = \textsf{aInit}_{\clientSet, \faultyClients}(\Gamma, \theta) \ \triangleq \ & \exists n. \ \exists i_1, ..., i_n. \ \exists {C}_{i_1}, ..., {C}_{i_n}.\ \clientSet \uplus \faultyClients  = \{ i_1, ..., i_n \} \land (\mathbb{W} = \textbf{with} \ \Gamma \textbf{ do } {C}_{i_1} \parallel ... \parallel {C}_{i_n})\\
			
%			\land\  (\forall t \in \clientSet.\ B_c(t)(\textit{faulty}) = \textbf{false} ) \land (\forall t \in \faultyClients.\ B_c(t)(\textit{faulty}) = \textbf{true} )\\
			
%			\land\  (\forall t \in \faultyClients.\ B_c(t)(\textit{F}_c) = \faultyClients) \\
			
			& \land \ (\forall t \in \clientSet.\ B_c(t)(\textit{faulty}) = \textbf{false} \land B_c(t)(\textit{Keys}) = \{t\} ) \\
			
			& \land \ (\forall t \in \faultyClients.\ B_c(t)(\textit{faulty}) = \textbf{true} \land B_c(t)(\textit{Keys}) = \faultyClients) \\
			
			& \land \ (\forall t \in \clientSet \uplus \faultyClients.\ (\mathbb{S}.\mathbb{K})(t) = \circ)  \\
			
			
			& {\land \ (\forall t \in \faultyClients.\ \exists {C}.\ {C}_t = C;\textbf{stop})}\\
			
		\end{array} $
	\end{definition}
	
	\textsf{aInit}意味着
	\begin{itemize}
		\item 正确的replica线程的初始程序只能有$ \textit{Stmt} $中定义的语句(即初始程序中没有\textbf{faulty-call}语句)
		
		\item faulty client的初始程序一定是某个程序$ C $与\textbf{stop}语句的串联. 用于标记faulty client的终止
		
%		\item faulty client初始程序中有$ \textbf{aftermath}(H) $,其中$ H $还是infinite的怎么办,能不能将aftermath和faulty-call统一起来?
		
	\end{itemize}
	
	
	\begin{definition}[\textbf{Legal Sequential Histories}]
	The set of all {\textit{legal sequential histories}} of protocol spec $ \Gamma $ and initial $ \theta $ \textit{w.r.t.} client set $ \clientSet $ and faulty client set $ \faultyClients $ is defined as follow:
	
	$ \absHistSet{\Gamma, \theta, \clientSet, \faultyClients} \ \triangleq \ \left\{ H \ \middle\vert  \begin{array}{ll}
		\exists \mathbb{W}, \mathbb{S}.\ (\mathbb{W}, \mathbb{S}) = \textsf{aInit}_{\clientSet, \faultyClients}(\Gamma, \theta) \ 
		\land \ \textsf{Hist}(\mathbb{W}, \mathbb{S}, H)  )
	\end{array}  \right\} $
	\end{definition}
	
	\vspace{0.2cm}
	
	
	\begin{comment}
	\begin{definition} [\textbf{Consensus Sequential Specification}]\ 
		
		$ \textsf{Gamma}(f) \ \triangleq \ \{ f  \rightsquigarrow \texttt{FP} \} $
		
		where 
		$$\begin{array}{ll}\			
			\texttt{FP}(pval)(\theta)\ \triangleq\ \left\{  
			\begin{array}{ll}
				(pval, \theta\{ \texttt{COBJ} \rightsquigarrow pval \}) & \textsf{if } (\theta(\texttt{COBJ}) = \texttt{undef}) \land (pval \neq \texttt{undef})\\
				(val, \theta) & \textsf{if } (\theta(\texttt{COBJ}) = val) \land (val \neq \texttt{undef})\\
				
				%				\textit{undefined} & \textit{o.w. }\\
			\end{array}
			\right.\\
		\end{array}$$
	
	\vspace{0.2cm}
	and $ \texttt{COBJ} $ is the program variable name of any {abstract} consensus object
		
	general initial object:	
		$$ \theta_\texttt{ini} \ \triangleq \ \{ \texttt{COBJ} \rightsquigarrow \texttt{undef} \} $$
	
	\end{definition}
	
	\end{comment}
	
%	物理地址:$\xi_i$
	
	
	
	
	\begin{comment}
	\begin{definition}[\textbf{Byznearizability}] {$ P \preceq_{\varphi} \Gamma $ iff}
		
		$ \begin{array}{ll}
		\forall B_o, \clientSet, \faultyClients, H.\  H \in \histSet{P, B_o, \clientSet, \faultyClients} \implies & (\exists H_c,H'.\ H_c \in \red{\textsf{completions}}(H) \land H' \in \absHistSet{\Gamma, \varphi(B_o), \clientSet, \faultyClients} 
		  \land (H_c \preceq_\textsf{lin}^{\clientSet, \faultyClients} H'))
		\end{array} $
	\end{definition}
	
	\begin{itemize}
		\item \red{由于$ H $可能infinite,我不是很明白completion怎么做。我想也许可以修改abstract machine 生成history的语义,使之前成总是对出现的function inv和res改为末尾(指线程t的分量的末尾)可以出现仅function inv事件的情形,这样就不需要用$ H $的complete(而直接用$ H $)来反映byz1和byz2}
	\end{itemize}
	\end{comment}

	\begin{definition}[\textbf{Byznearizability}] {$ P \preceq_{\varphi} \Gamma $ iff}
		
		$ \begin{array}{ll}
			\forall B_o, \clientSet, \faultyClients, H.\  H \in \histSet{P, B_o, \clientSet, \faultyClients} \implies & (\exists H'.\ H' \in \absHistSet{\Gamma, \varphi(B_o), \clientSet, \faultyClients} 
			\land (H \preceq_\textsf{lin}^{\clientSet, \faultyClients} H'))
		\end{array} $
	\end{definition}
	
	
	\begin{itemize}
		\item 由于这里的history是由$ \textsf{Hist}(W,S,H) $定义,若history on 某thread的projection是有限长的,则我认为该线程不会有pending invocation, 若无限长,就也不用考虑pending的问题了。所以我没有定义completion
	\end{itemize}
	
	\pagebreak
	
	
\end{small}


\end{document}