
% ===============================================
% MATH 29: Discrete Mathematics           Fall 2018
% hw_template.tex
% ===============================================

% -------------------------------------------------------------------------
% You can ignore this preamble. Go on
% down to the section that says "START HERE" 
% -------------------------------------------------------------------------

\documentclass{ctexart}
\usepackage{xcolor}

\usepackage[margin=1.5in]{geometry} % Please keep the margins at 1.5 so that there is space for grader comments.
\usepackage{amsmath,amsthm,amssymb,hyperref}

\newcommand{\R}{\mathbf{R}}  
\newcommand{\Z}{\mathbf{Z}}
\newcommand{\N}{\mathbf{N}}
\newcommand{\Q}{\mathbf{Q}}

\theoremstyle{definition}
\newtheorem{definition}{Definition}[section]
% \newtheorem{theorem}{Theorem}[section]
% \newtheorem{corollary}{Corollary}[theorem]
% \newtheorem{lemma}[theorem]{Lemma}

\newenvironment{theorem}[2][Theorem]{\begin{trivlist}
\item[\hskip \labelsep {\bfseries #1}\hskip \labelsep {\bfseries #2.}]}{\end{trivlist}}
\newenvironment{lemma}[2][Lemma]{\begin{trivlist}
\item[\hskip \labelsep {\bfseries #1}\hskip \labelsep {\bfseries #2.}]}{\end{trivlist}}
\newenvironment{claim}[2][Claim]{\begin{trivlist}
\item[\hskip \labelsep {\bfseries #1}\hskip \labelsep {\bfseries #2.}]}{\end{trivlist}}
\newenvironment{problem}[2][Problem]{\begin{trivlist}
\item[\hskip \labelsep {\bfseries #1}\hskip \labelsep {\bfseries #2.}]}{\end{trivlist}}
\newenvironment{proposition}[2][Proposition]{\begin{trivlist}
\item[\hskip \labelsep {\bfseries #1}\hskip \labelsep {\bfseries #2.}]}{\end{trivlist}}
\newenvironment{corollary}[2][Corollary]{\begin{trivlist}
\item[\hskip \labelsep {\bfseries #1}\hskip \labelsep {\bfseries #2.}]}{\end{trivlist}}

\newenvironment{solution}{\begin{proof}[Solution]}{\end{proof}}

\begin{document}

\large % please keep the text at this size for ease of reading.
\linespread{1.5} % please keep 1.5 line spacing so that there is space for grader comments.

% ------------------------------------------ %
%                 START HERE             %
% ------------------------------------------ %

% {\Large Week 2 proofs % Replace with appropriate number -- keep this THE SAME even when you revise and resubmit.
% \hfill  Math 29, Discrete Mathematics}

\begin{center}
% {\Large Author's Name} % Replace "Author's Name" with your name
{\Large Hoare soundness proof}
\end{center}
\vspace{0.05in}

% -----------------------------------------------------
% The following two environments (claim, proof) are
% where you will enter the statement and proof of your
% first problem for this assignment.
%
% In the theorem environment, you can replace the word
% "theorem" in the \begin and \end commands with
% "theorem", "proposition", "lemma", etc., depending on
% what you are proving. 
% -----------------------------------------------------


\begin{claim}{Soundness for partial correctness}
if $\vdash\{p\}c\{q\}$, then $\models\{p\}c\{q\}$.  
\end{claim}

% \theoremstyle{definition}
\begin{definition}{$\models\{p\}c\{q\}$}
\newline
$\models\{p\}c\{q\} \iff \forall \sigma,\sigma'.(\sigma\models p)\wedge((c,\sigma)\longrightarrow^*(skip, \sigma'))\Rightarrow(\sigma'\models q)$
\end{definition}

\begin{definition}{$\vdash\{p\}c\{q\}$}
    \newline
    There exists a proof or derivation for $\{p\}c\{q\}$.
\end{definition}

用自然语言描述,我们要证明满足p的所有状态$\sigma$,在执行语句c后(按照语义定义进行evaluate),或者不终止,或者终态满足$q$.

首先我们希望描述对某单一初始状态$\sigma$,相对于语句c和终态q的“soundness”,也就是课件中定义的\textit{Safe($c,\sigma,q$)}. 这一步可以理解为设立一个中间目标替换$\models\{p\}c\{q\}$,去掉$\sigma$前的全称量词
,以进行inductive的证明。如果对于某个p,所有符合它的状态都满足这种描述,那么就是符合语义的了。

\begin{definition}{$Safe^n(c,\sigma,q$)}
    \newline $Safe^0(c,\sigma,q)$ always holds;
    \newline $Safe^{n+1}(c,\sigma,q)$ holds iff (a) c = skip and $\sigma \models q$; or (b) 
    \color{red}{there exists $c'$ and $\sigma'$ such that $(c, \sigma)\longrightarrow(c', \sigma')$} and $Safe^n(c',\sigma',q)$.
    \newline 
    \color{black} We say $Safe(c,\sigma,q)$ iff $Safe^n(c,\sigma,q)$ holds for all n.
    \newline 用自然语言描述:某一个状态$\sigma$对于某指令c和某谓词q是Safe的,当它存在某一条不终止的执行方式,或者一条终止的、且终态满足q的执行方式。(即允许歧义性)
\end{definition}


\begin{lemma}{1}
$Safe(c,\sigma,q) \Rightarrow ((c,\sigma)\longrightarrow^*(skip,\sigma'))\Rightarrow(\sigma'\models q)$
\end{lemma}


\begin{proof}
我们对Safe的定义是直接为Lemma1服务的,它的证明也应是obvious的. 
Follow by lemma 1.1 and lemma 1.2. 但我认为lemma1.2不是trival的(要求没有二义性就好了).
\end{proof}

\begin{lemma}{1.1 Progress}
    If $Safe(c,\sigma,q)$, then either c is skip, or there exists $c'$ and $\sigma'$ such that 
    $(c, \sigma)\longrightarrow(c', \sigma')$. 
\end{lemma}

\begin{proof}
    Trival.
\end{proof}


\begin{lemma}{1.2 Preservation} 
    If $Safe(c,\sigma,q)$ and $(c,\sigma)\longrightarrow(c', \sigma')$, then 
    $Safe(c', \sigma', q)$.
\end{lemma}
\begin{proof}
    并不能看出这一步是trival的,因为它要求语义是没有二义性的。
\end{proof}

\begin{lemma}{2}
    $\vdash\{p\}c\{q\} \Rightarrow \forall\sigma.(\sigma\models p)\Rightarrow Safe(c,\sigma,q)$
\end{lemma}

\begin{proof}
    Proof by induction over the derivation of $\vdash\{p\}c\{q\}$.
\end{proof}

\begin{lemma}{3. Lead to Soundness}
    $lemma1 \wedge lemma2 \Rightarrow\;\vdash\{p\}c\{q\} \Rightarrow ( \forall \sigma,\sigma'.((\sigma\models p)\wedge((c,\sigma)\longrightarrow^*(skip, \sigma')))\Rightarrow(\sigma'\models q))$
\end{lemma}

\begin{proof}
    Trival.
\end{proof}

\end{document}

